
\documentclass[11pt]{article}

\newcommand{\issolution}{true}
\newcommand{\declarecommand}[1]{\providecommand{#1}{}\renewcommand{#1}}

\input{preamble.tex}
%% Macros for 15-150 assignments

\newcommand{\printdue}[1]{\vspace{-0.5cm}{\Large{\textbf{\hl{Due:} #1}}}}

\newcommand{\hint}[1]{\footnote{ {\it Hint:} #1}}

%% Task and solution environments

\newcounter{taskCounter}
\setcounter{taskCounter}{0}

\newcounter{taskBonusCounter}
\setcounter{taskBonusCounter}{0}

\newcounter{pointsCounter}
\setcounter{pointsCounter}{0}

\newcounter{pointsBonusCounter}
\setcounter{pointsBonusCounter}{0}

% Task numbering variables resets to zero each time section changes
\newcounter{taskNum}[section]
\setcounter{taskNum}{0}

% Written tasks
\newenvironment{task}[1]
{
  \addtocounter{taskCounter}{1}
  \addtocounter{pointsCounter}{#1}
  \addtocounter{taskNum}{1}
  \begin{framed}
  \noindent
  \textbf{Task \arabic{section}.\arabic{taskNum}} (#1 points)
}
{
  \end{framed}
}

% Code tasks (do not need answers in the pdf)
\newenvironment{codetask}[1]
{
  \addtocounter{taskCounter}{1}
  \addtocounter{pointsCounter}{#1}
  \addtocounter{taskNum}{1}
  \begin{framed}
  \noindent
  \textbf{Coding Task \arabic{section}.\arabic{taskNum}} (#1 points)
}
{
  \end{framed}
}

% Bonus tasks are highlighted
\newenvironment{bonustask}[1]
{
  \addtocounter{taskBonusCounter}{1}
  \addtocounter{pointsBonusCounter}{#1}
  \addtocounter{taskNum}{1}
  \begin{framed}
  \noindent
  \colorbox{yellow}{%
    \textbf{Bonus Task \arabic{section}.\arabic{taskNum}} (#1 points)
  }
}
{
  \end{framed}
}

% Bonus code tasks (do not need answers in the pdf)
\newenvironment{bonuscodetask}[1]
{
  \addtocounter{taskBonusCounter}{1}
  \addtocounter{pointsBonusCounter}{#1}
  \addtocounter{taskNum}{1}
  \begin{framed}
  \noindent
  \colorbox{yellow}{%
    \textbf{Bonus Coding Task \arabic{section}.\arabic{taskNum}} (#1 points)
  }
}
{
  \end{framed}
}

% Lab tasks
\newenvironment{labtask}[1]
{
  \addtocounter{taskCounter}{1}
  \addtocounter{pointsCounter}{#1}
  \addtocounter{taskNum}{1}
  \begin{framed}
  \noindent
  \textbf{Task \arabic{section}.\arabic{taskNum}} (#1 points)
}
{
  \end{framed}
}


\AtEndDocument{\clearpage
  \typeout{^^J^^J^^J******** Assignment summary ********}
  \typeout{^^J-- \arabic{taskCounter} tasks worth \arabic{pointsCounter} points}
  \ifthenelse{\value{taskBonusCounter} > 0}
  {
    \typeout{^^J--\arabic{taskBonusCounter} bonus tasks worth \arabic{pointsBonusCounter} bonus points}
  }{}
  \typeout{^^J^^J************************************^^J^^J^^J}
}


\definecolor{solutioncolor}{rgb}{0.5, 0.0, 0.0}

% Reference solution (colored)
\ifthenelse{\equal{\issolution}{true}}
{
  \newenvironment{refsol}
  {%
    \noindent%
    \fbox{\textcolor{solutioncolor}{\bf Solution \arabic{section}.\arabic{taskNum}}} 
    \begingroup\color{solutioncolor}
  }
  {
    \endgroup
  }
}
{
  \excludecomment{refsol}
}

% Student solution (no color)
\ifthenelse{\equal{\issolution}{true}}
{
  \newenvironment{solution}
  {%
    \noindent%
    {\bf Solution \arabic{section}.\arabic{taskNum}}
  }
  {
  }
  \newenvironment{labsol}[1]
  {%
    \begin{center}
    \begin{tabular}{|p{0.97\textwidth}|}
    \hline
    {\bf Solution \arabic{section}.\arabic{taskNum}} \\[#1]
  }
  {
    \\\hline
    \end{tabular}
    \end{center}
  }
}
{
  \excludecomment{solution}
}



\title{\textbf{15-150 Fall 2024\\Homework 03}}
\author{YOUR NAME HERE -- \texttt{YOUR ANDREWID HERE}}

\begin{document}
\maketitle

\declarecommand{\nil}{\mathit{nil}}
\declarecommand{\op}{mult}
\declarecommand{\opMf}[2]{#1 \times #2}
\declarecommand{\opCf}[2]{\smle{#1 * #2}}
\declarecommand{\opM}{\mathit{\op}}
\declarecommand{\opMTail}{\mathit{{\op}Tail}}
\declarecommand{\opMTailAux}{\mathit{{\op}Tail'}}
\declarecommand{\opC}{\smle{mult}}
\declarecommand{\opCTail}{\smle{multTail}}
\declarecommand{\opCTailAux}{\smle{multTail'}}
\section{Straight Recursion versus Tail Recursion}

\subsection{Multiplying a list}

\begin{task}{2} %
Give the corresponding tail recursive definition, call it
$\mathit{multTail}$.
\end{task}

\begin{solution}
TYPE YOUR SOLUTION HERE
\end{solution}


\addtocounter{taskNum}{1}
\begin{task}{3} %
Write the recurrence relation for the work of \smle{mult}, deduce an upper-bound
approximation in closed form, and determine a tight big-O class for it.  Do
the same for \smle{multTail}.  Which one is more efficient in practice?  Feel
free to refer to calculations seen in class, but cite each occurrence.
\end{task}

\begin{solution}
TYPE YOUR SOLUTION HERE
\end{solution}


\declarecommand{\Wa}{W_{\smle{mult}}}
\declarecommand{\Wb}{W_{\smle{multTail}}}
\declarecommand{\Wc}{W_{\smle{multTail'}}}
\begin{task}{6} %
Show that the inductive function definitions for \smle{mult} and
\smle{multTail} compute the same mathematical function by proving the
following property:

\begin{property}
For all \smle{l : int list}, \smle{mult l} $\cong$ \smle{multTail l}
\end{property}

\noindent
If this proof makes use of an auxiliary lemma, state it and prove it.
\end{task}

\begin{solution}
TYPE YOUR SOLUTION HERE
\end{solution}


\declarecommand{\Leaf}{\mathit{Leaf}}
\declarecommand{\Node}{\mathit{Node}}
\declarecommand{\treeOp}{\mathit{treeMin}}
\declarecommand{\op}[2]{\min\{#1,#2\}}
\subsection{Minimum of a Tree}

\begin{task}{2} %
Define the function $\treeOp: \mathbb{T} \rightarrow \mathbb{Z}$ that returns
the smallest number in the tree given as input.
\end{task}

\begin{solution}
TYPE YOUR SOLUTION HERE
\end{solution}


\addtocounter{taskNum}{1}
\begin{task}{1} %
Is it possible to give a tail-recursive implementation of $\treeOp$?  If your
answer is ``yes'', give the code for it.  If your answer is ``no'', explain why.
\end{task}

\begin{solution}
TYPE YOUR SOLUTION HERE
\end{solution}


\section{Prime factorization}

\subsection{Let's factorize}

\addtocounter{taskNum}{1}
\addtocounter{taskNum}{1}
\subsection{Certificates}

\addtocounter{taskNum}{1}
\addtocounter{taskNum}{1}
\subsection{Prime, or not}

\addtocounter{taskNum}{1}
\addtocounter{taskNum}{1}
\addtocounter{taskNum}{1}
\section{Conway's Lost Cosmological Theorem}

\subsection{Look and Say}

\subsection{\ldots and Code too}

\addtocounter{taskNum}{1}
\addtocounter{taskNum}{1}
\addtocounter{taskNum}{1}
\subsection{Cultural Aside}

\end{document}
