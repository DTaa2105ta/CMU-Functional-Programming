
\documentclass[11pt]{article}

\newcommand{\issolution}{true}
\newcommand{\declarecommand}[1]{\providecommand{#1}{}\renewcommand{#1}}

%% Packages and Macros for 15-150

\usepackage{graphicx}
\usepackage{amsmath}
\usepackage{amssymb}
\usepackage{amsthm}
\usepackage{stmaryrd}
\usepackage[colorlinks=true,breaklinks]{hyperref}
\usepackage[pdftex,dvipsnames]{xcolor} 
\usepackage{cancel}
\usepackage{framed}
\usepackage{comment}
\usepackage[short]{datetime}
\usepackage{caption}
\usepackage{subcaption}
\usepackage{etoolbox}
\usepackage{xparse}

\usepackage{multicol}
\setlength{\columnseprule}{1pt}

\usepackage{soul}
\sethlcolor{yellow}

\usepackage[margin=2cm]{geometry}

\newtheorem{remark}{Remark}
\newtheorem{theorem}{Theorem}
\newtheorem{lemma}{Lemma}
\newtheorem{definition}{Definition}
\newtheorem{corollary}{Corollary}
\newtheorem{property}[theorem]{Property}
\newtheorem{proposition}[theorem]{Proposition}

% For quotes
\usepackage{csquotes}

% symbols
\newrobustcmd{\eeq}{\ensuremath{\cong}}

% Boxes used in proof template
\usepackage[colorinlistoftodos,prependcaption,textsize=footnotesize]{todonotes}
\newcommand{\expl}[2]{\todo[inline,linecolor=#1,backgroundcolor=#1!25,bordercolor=#1]{#2}}

\newcommand{\think}[1]{%
  \begin{framed}
    {\color{red}\textbf{Brain teaser:}} #1
  \end{framed}
}
\newcommand{\exercise}[1]{%
  \begin{framed}
    {\color{red}\textbf{Exercise:}} #1
  \end{framed}
}

% Markers for quantified variables (soundness and completeness
% lecture)
\newcommand{\eigenvar}[1]{\colorbox{red!50}{#1}}
\newcommand{\term}[1]{\colorbox{green!50}{#1}}

% Listings setting for SML code
\usepackage{listings}
\definecolor{eclipseBlue}{RGB}{42,0.0,255}
\definecolor{eclipseGreen}{RGB}{63,127,95}
\lstdefinelanguage{SML}{
  basicstyle=\small\ttfamily,
  captionpos=b,
  tabsize=2,
  columns=fixed,
  breaklines=true,
  showstringspaces=false,
  mathescape=true,
  %frame=l,
  numbers=none,
  upquote=true,
  numberstyle=\small\ttfamily,
  morekeywords= {
    EQUAL, GREATER, LESS, NONE, SOME, 
    abstraction, abstype, and, andalso, 
    array, as, before, bool, case, char, 
    datatype, do, else, end, eqtype, 
    exception, exn, false, fn, fun, 
    functor, handle, if, in, include, 
    infix, infixr, int, let, list, local, 
    nil, nonfix, not, o, of, op, open, 
    option, orelse, overload, print, 
    raise, real, rec, ref, sharing, sig, 
    signature, string, struct, structure, 
    substring, then, true, type, unit, val, 
    vector, where, while, with, withtype, word
  },
  morestring=[b]",
  morecomment=[s]{(*}{*)},
  stringstyle=\color{black},
  identifierstyle=\color{eclipseBlue},
  keywordstyle=\color{red},
  commentstyle=\color{eclipseGreen},
  escapeinside={!?}{?!}
}

% Code block
\lstnewenvironment{sml}[1][]{\lstset{language=SML,#1}}{}
% Inline code
\def\smle{\lstinline[language=SML]}

% So that smle works inside math mode
\usepackage{letltxmacro}
\newcommand*{\SavedLstInline}{}
\LetLtxMacro\SavedLstInline\smle
\DeclareRobustCommand*{\smle}{%
  \ifmmode
    \let\SavedBGroup\bgroup
    \def\bgroup{%
      \let\bgroup\SavedBGroup
      \hbox\bgroup
    }%
  \fi
  \SavedLstInline
}

% For red-black trees
\usepackage{tikz}
\usetikzlibrary{arrows}
\usetikzlibrary{shapes.geometric}
\tikzset{
  treenode/.style = {align=center, inner sep=0pt, text centered,
    font=\sffamily},
  arn_n/.style = {treenode, circle, white, font=\sffamily\bfseries, draw=black,
    fill=black, text width=1.5em},% arbre rouge noir, noeud noir
  arn_r/.style = {treenode, circle, red, draw=red,
    text width=1.5em, very thick},% arbre rouge noir, noeud rouge
  arn_x/.style = {treenode, rectangle, %draw=black,
    minimum width=0.5em, minimum height=0.5em},% arbre rouge noir, nil
  arn_t/.style = {treenode, regular polygon, regular polygon sides=3,
    fill=Sepia!20, draw=black, minimum size=4.0em}% arbre rouge noir, tree
}

%% Macros for 15-150 assignments

\newcommand{\printdue}[1]{\vspace{-0.5cm}{\Large{\textbf{\hl{Due:} #1}}}}

\newcommand{\hint}[1]{\footnote{ {\it Hint:} #1}}

%% Task and solution environments

\newcounter{taskCounter}
\setcounter{taskCounter}{0}

\newcounter{taskBonusCounter}
\setcounter{taskBonusCounter}{0}

\newcounter{pointsCounter}
\setcounter{pointsCounter}{0}

\newcounter{pointsBonusCounter}
\setcounter{pointsBonusCounter}{0}

% Task numbering variables resets to zero each time section changes
\newcounter{taskNum}[section]
\setcounter{taskNum}{0}

% Written tasks
\newenvironment{task}[1]
{
  \addtocounter{taskCounter}{1}
  \addtocounter{pointsCounter}{#1}
  \addtocounter{taskNum}{1}
  \begin{framed}
  \noindent
  \textbf{Task \arabic{section}.\arabic{taskNum}} (#1 points)
}
{
  \end{framed}
}

% Code tasks (do not need answers in the pdf)
\newenvironment{codetask}[1]
{
  \addtocounter{taskCounter}{1}
  \addtocounter{pointsCounter}{#1}
  \addtocounter{taskNum}{1}
  \begin{framed}
  \noindent
  \textbf{Coding Task \arabic{section}.\arabic{taskNum}} (#1 points)
}
{
  \end{framed}
}

% Bonus tasks are highlighted
\newenvironment{bonustask}[1]
{
  \addtocounter{taskBonusCounter}{1}
  \addtocounter{pointsBonusCounter}{#1}
  \addtocounter{taskNum}{1}
  \begin{framed}
  \noindent
  \colorbox{yellow}{%
    \textbf{Bonus Task \arabic{section}.\arabic{taskNum}} (#1 points)
  }
}
{
  \end{framed}
}

% Bonus code tasks (do not need answers in the pdf)
\newenvironment{bonuscodetask}[1]
{
  \addtocounter{taskBonusCounter}{1}
  \addtocounter{pointsBonusCounter}{#1}
  \addtocounter{taskNum}{1}
  \begin{framed}
  \noindent
  \colorbox{yellow}{%
    \textbf{Bonus Coding Task \arabic{section}.\arabic{taskNum}} (#1 points)
  }
}
{
  \end{framed}
}

% Lab tasks
\newenvironment{labtask}[1]
{
  \addtocounter{taskCounter}{1}
  \addtocounter{pointsCounter}{#1}
  \addtocounter{taskNum}{1}
  \begin{framed}
  \noindent
  \textbf{Task \arabic{section}.\arabic{taskNum}} (#1 points)
}
{
  \end{framed}
}


\AtEndDocument{\clearpage
  \typeout{^^J^^J^^J******** Assignment summary ********}
  \typeout{^^J-- \arabic{taskCounter} tasks worth \arabic{pointsCounter} points}
  \ifthenelse{\value{taskBonusCounter} > 0}
  {
    \typeout{^^J--\arabic{taskBonusCounter} bonus tasks worth \arabic{pointsBonusCounter} bonus points}
  }{}
  \typeout{^^J^^J************************************^^J^^J^^J}
}


\definecolor{solutioncolor}{rgb}{0.5, 0.0, 0.0}

% Reference solution (colored)
\ifthenelse{\equal{\issolution}{true}}
{
  \newenvironment{refsol}
  {%
    \noindent%
    \fbox{\textcolor{solutioncolor}{\bf Solution \arabic{section}.\arabic{taskNum}}} 
    \begingroup\color{solutioncolor}
  }
  {
    \endgroup
  }
}
{
  \excludecomment{refsol}
}

% Student solution (no color)
\ifthenelse{\equal{\issolution}{true}}
{
  \newenvironment{solution}
  {%
    \noindent%
    {\bf Solution \arabic{section}.\arabic{taskNum}}
  }
  {
  }
  \newenvironment{labsol}[1]
  {%
    \begin{center}
    \begin{tabular}{|p{0.97\textwidth}|}
    \hline
    {\bf Solution \arabic{section}.\arabic{taskNum}} \\[#1]
  }
  {
    \\\hline
    \end{tabular}
    \end{center}
  }
}
{
  \excludecomment{solution}
}



\title{\textbf{15-150 Fall 2024\\Homework 05}}
\author{YOUR NAME HERE -- \texttt{YOUR ANDREWID HERE}}

\begin{document}
\maketitle

\section{SML in Action}

\subsection{Type Inference}

\begin{task}{2} %
Carry out type inference on the following function:
\begin{sml}[numbers=left]
fun woof ([], puppy) = puppy
  | woof (sooo::CUTE, puppy) = 4.0 + woof(CUTE, puppy)
\end{sml}
\end{task}

\begin{solution}
TYPE YOUR SOLUTION HERE
\end{solution}


\begin{task}{2} %
Carry out type inference on the following function:
\begin{sml}[numbers=left]
fun f (empty, n) = nil
  | f (node(L,x,R),n) = node(f(L,n+1), (x,n), f(R,n+1))
\end{sml}
\end{task}

\begin{solution}
TYPE YOUR SOLUTION HERE
\end{solution}


\begin{task}{2} %
Carry out type inference on the following function:
\begin{sml}[numbers=left]
fun f x = f (f (x+1))
\end{sml}
\end{task}

\begin{solution}
TYPE YOUR SOLUTION HERE
\end{solution}


\subsection{Scope}

\begin{task}{6} %
Consider the following code fragment:
\begin{sml}[numbers=left]
val x : int = 5
val y : real = 2.0
val temp : real = real (x - 1)
fun generate (x : int, y : real, z : real) : int =
  let
    val g : real =
      let
        val x : int = 3
        val z : real = z * y
        val k : real = temp * (real x) 
        val a : int = 38
        val y : real = k * y
      in
        z * y + (real a) + k
      end
  in
    x + trunc g
  end

val z = generate (x, y, temp)
\end{sml}

\begin{enumerate}
\item%
  What value gets substituted for the variable \smle{x} on line (10)?  Briefly
  explain why.  What is its type?
\item%
  What value gets substituted for the variable \smle{k} on line (12)?  Briefly
  explain why.  What is its type?
\item%
  What value gets substituted for the variable \smle{x} on line (17)?  Briefly
  explain why.  What is its type?
\item%
  What value does the expression \smle{generate (x, y, temp)} evaluate to on
  line (20)?
\end{enumerate}
\end{task}

\begin{solution}
TYPE YOUR SOLUTION HERE
\end{solution}


\begin{task}{7} %
Consider the following code fragment:
\begin{sml}[numbers=left]
val x = ~1
fun f x =
    let
      val y = x
      val x = 90
      fun f x = y + 9
      val x = 50
    in
      f x
    end
val x = f x
\end{sml}

During the evaluation of \smle{f x} on line (11),
\begin{enumerate}
\item%
  What value gets substituted for the variable \smle{y} on line (4)?  Briefly
  explain why.  What is its type?
\item%
  What value gets substituted for the variable \smle{x} on line (9)?  Briefly
  explain why.  What is its type?
\item%
  To what is the identifier \smle{f} bound on line (9)?
\item%
  What value gets substituted for the rightmost occurence of the variable
  \smle{x} on line (11)?  Briefly explain why.  What is its type?
\item%
  What value gets substituted for the leftmost occurence of the variable
  \smle{x} on line (11)?  Briefly explain why.  What is its type?
\end{enumerate}
\end{task}

\begin{solution}
TYPE YOUR SOLUTION HERE
\end{solution}


\begin{task}{6} %
Consider the following code fragment:
\begin{sml}[numbers=left]
val x = 13
val x =
    let
      val x = 14
      val x = (case x mod 4
                of 1 => "x"
                 | 2 => "o") ^
              let
                val x = 9
                val x = "x"
              in
                x
              end
    in
      x ^ "o"
    end
\end{sml}

\begin{enumerate}
\item%
  What value gets substituted for the leftmost occurrence of the variable
  \smle{x} on line (5)?  Briefly explain why.  What is its type?
\item%
  What value gets substituted for the variable \smle{x} on line (12)?  Briefly
  explain why.  What is its type?
\item%
  What value gets substituted for the variable \smle{x} on line (15)?  Briefly
  explain why.  What is its type?
\item%
  What would change if line 1 were changed to \smle{val x = 12}? Why?
\item%
  What would change if line 4 were changed to \smle{val x = 12}? Why?
\end{enumerate}
\end{task}

\begin{solution}
TYPE YOUR SOLUTION HERE
\end{solution}


\section{Delimiters}

\subsection{Nested Parentheses}

\addtocounter{taskNum}{1}
\subsection{Parentheses Trees}

\addtocounter{taskNum}{1}
\addtocounter{taskNum}{1}
\addtocounter{taskNum}{1}
\subsection{Beyond Parentheses}

\addtocounter{taskNum}{1}
\addtocounter{taskNum}{1}
\addtocounter{taskNum}{1}
\addtocounter{taskNum}{1}
\section{Common Subsequences}

\begin{task}{3}
Write down a mathematical inductive definition for a function that computes the
longest common subsequence of two sequences based on the properties above.
\end{task}

\begin{solution}
TYPE YOUR SOLUTION HERE
\end{solution}


\addtocounter{taskNum}{1}
\addtocounter{taskNum}{1}
\end{document}
