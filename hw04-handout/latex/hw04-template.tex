
\documentclass[11pt]{article}

\newcommand{\issolution}{true}
\newcommand{\declarecommand}[1]{\providecommand{#1}{}\renewcommand{#1}}

\input{preamble.tex}
%% Macros for 15-150 assignments

\newcommand{\printdue}[1]{\vspace{-0.5cm}{\Large{\textbf{\hl{Due:} #1}}}}

\newcommand{\hint}[1]{\footnote{ {\it Hint:} #1}}

%% Task and solution environments

\newcounter{taskCounter}
\setcounter{taskCounter}{0}

\newcounter{taskBonusCounter}
\setcounter{taskBonusCounter}{0}

\newcounter{pointsCounter}
\setcounter{pointsCounter}{0}

\newcounter{pointsBonusCounter}
\setcounter{pointsBonusCounter}{0}

% Task numbering variables resets to zero each time section changes
\newcounter{taskNum}[section]
\setcounter{taskNum}{0}

% Written tasks
\newenvironment{task}[1]
{
  \addtocounter{taskCounter}{1}
  \addtocounter{pointsCounter}{#1}
  \addtocounter{taskNum}{1}
  \begin{framed}
  \noindent
  \textbf{Task \arabic{section}.\arabic{taskNum}} (#1 points)
}
{
  \end{framed}
}

% Code tasks (do not need answers in the pdf)
\newenvironment{codetask}[1]
{
  \addtocounter{taskCounter}{1}
  \addtocounter{pointsCounter}{#1}
  \addtocounter{taskNum}{1}
  \begin{framed}
  \noindent
  \textbf{Coding Task \arabic{section}.\arabic{taskNum}} (#1 points)
}
{
  \end{framed}
}

% Bonus tasks are highlighted
\newenvironment{bonustask}[1]
{
  \addtocounter{taskBonusCounter}{1}
  \addtocounter{pointsBonusCounter}{#1}
  \addtocounter{taskNum}{1}
  \begin{framed}
  \noindent
  \colorbox{yellow}{%
    \textbf{Bonus Task \arabic{section}.\arabic{taskNum}} (#1 points)
  }
}
{
  \end{framed}
}

% Bonus code tasks (do not need answers in the pdf)
\newenvironment{bonuscodetask}[1]
{
  \addtocounter{taskBonusCounter}{1}
  \addtocounter{pointsBonusCounter}{#1}
  \addtocounter{taskNum}{1}
  \begin{framed}
  \noindent
  \colorbox{yellow}{%
    \textbf{Bonus Coding Task \arabic{section}.\arabic{taskNum}} (#1 points)
  }
}
{
  \end{framed}
}

% Lab tasks
\newenvironment{labtask}[1]
{
  \addtocounter{taskCounter}{1}
  \addtocounter{pointsCounter}{#1}
  \addtocounter{taskNum}{1}
  \begin{framed}
  \noindent
  \textbf{Task \arabic{section}.\arabic{taskNum}} (#1 points)
}
{
  \end{framed}
}


\AtEndDocument{\clearpage
  \typeout{^^J^^J^^J******** Assignment summary ********}
  \typeout{^^J-- \arabic{taskCounter} tasks worth \arabic{pointsCounter} points}
  \ifthenelse{\value{taskBonusCounter} > 0}
  {
    \typeout{^^J--\arabic{taskBonusCounter} bonus tasks worth \arabic{pointsBonusCounter} bonus points}
  }{}
  \typeout{^^J^^J************************************^^J^^J^^J}
}


\definecolor{solutioncolor}{rgb}{0.5, 0.0, 0.0}

% Reference solution (colored)
\ifthenelse{\equal{\issolution}{true}}
{
  \newenvironment{refsol}
  {%
    \noindent%
    \fbox{\textcolor{solutioncolor}{\bf Solution \arabic{section}.\arabic{taskNum}}} 
    \begingroup\color{solutioncolor}
  }
  {
    \endgroup
  }
}
{
  \excludecomment{refsol}
}

% Student solution (no color)
\ifthenelse{\equal{\issolution}{true}}
{
  \newenvironment{solution}
  {%
    \noindent%
    {\bf Solution \arabic{section}.\arabic{taskNum}}
  }
  {
  }
  \newenvironment{labsol}[1]
  {%
    \begin{center}
    \begin{tabular}{|p{0.97\textwidth}|}
    \hline
    {\bf Solution \arabic{section}.\arabic{taskNum}} \\[#1]
  }
  {
    \\\hline
    \end{tabular}
    \end{center}
  }
}
{
  \excludecomment{solution}
}



\title{\textbf{15-150 Fall 2024\\Homework 04}}
\author{YOUR NAME HERE -- \texttt{YOUR ANDREWID HERE}}

\begin{document}
\maketitle

\section{Insertion Sort}

\subsection{Insertion Sort on Lists}

\subsection{Sorting Trees by Insertion}

\begin{task}{6} %
Write the recurrence relation for the work of \smle{Insert} in terms of the
height of the input tree, deduce an upper-bound approximation in closed form,
and determine a tight big-O class for it.  Use this to identify the best and
worst cases measured in terms of the number of nodes in the tree, and give
their big-O classes.  Do the same for its span.
\end{task}

\begin{solution}
TYPE YOUR SOLUTION HERE
\end{solution}


\declarecommand{\Wa}{W_{\smle{Insert}}}
\declarecommand{\Sa}{S_{\smle{Insert}}}
\addtocounter{taskNum}{1}
\begin{task}{10} %
Write the recurrence relation for the work of \smle{ILsort} in terms of the
number of nodes in the tree.  For both the best case and the worst case,
deduce an upper-bound approximation in closed form, and determine a tight
big-O class for it.  Do the same for its span.
\end{task}

\begin{solution}
TYPE YOUR SOLUTION HERE
\end{solution}


\declarecommand{\Wa}{W_{\smle{Insert}}}
\declarecommand{\Wb}{W_{\smle{InsList}}}
\declarecommand{\Wc}{W_{\smle{ILsort}}}
\declarecommand{\Wd}{W_{\smle{inorder}}}
\declarecommand{\Sa}{S_{\smle{Insert}}}
\declarecommand{\Sb}{S_{\smle{InsList}}}
\declarecommand{\Sc}{S_{\smle{ILsort}}}
\declarecommand{\Sd}{S_{\smle{inorder}}}
\begin{task}{2} %
Chances are that your code does not prevent the worst case scenario from
happening.  How would you modify it so that the best case is always realized.
Explain why.  [You are not required to modify your code in this way, although
you are welcome to.]
\end{task}

\begin{solution}
TYPE YOUR SOLUTION HERE
\end{solution}


\declarecommand{\Wa}{W_{\smle{rebalance}}}
\declarecommand{\Sa}{S_{\smle{rebalance}}}
\subsection{Insertion Sort for Trees}

\addtocounter{taskNum}{1}
\section{Balanced Trees}

\addtocounter{taskNum}{1}
\addtocounter{taskNum}{1}
\addtocounter{taskNum}{1}
\addtocounter{taskNum}{1}
\begin{task}{2} %
  Give a recurrence that describes the work $W_{\smle{splitN}}(d)$ of
  \smle{splitN} in terms of the \emph{height} $d$ of the input tree.  Give a
  tight big-O bound for $W_{\smle{splitN}}(d)$.
\end{task}

\begin{solution}
TYPE YOUR SOLUTION HERE
\end{solution}


\declarecommand{\Wa}{W_{\smle!splitN!}}
\declarecommand{\Wb}{W_{\smle!leftmost!}}
\declarecommand{\Wc}{W_{\smle!halves!}}
\declarecommand{\Wd}{W_{\smle!rebalance!}}
\begin{task}{2} %
  Give a recurrence that describes the work $W_{\smle{leftmost}}(d)$ of
  \smle{leftmost} in terms of the \emph{height} $d$ of the input tree.  Give a
  tight big-O bound for $W_{\smle{leftmost}}(d)$.
\end{task}

\begin{solution}
TYPE YOUR SOLUTION HERE
\end{solution}


\declarecommand{\Wa}{W_{\smle!splitN!}}
\declarecommand{\Wb}{W_{\smle!leftmost!}}
\declarecommand{\Wc}{W_{\smle!halves!}}
\declarecommand{\Wd}{W_{\smle!rebalance!}}
\begin{task}{2} %
  Give a recurrence that describes the work $W_{\smle{halves}}(d)$ of
  \smle{halves} in terms of the \emph{height} of the input tree.  Give a tight
  big-O bound for $W_{\smle{halves}}(d)$.
\end{task}

\begin{solution}
TYPE YOUR SOLUTION HERE
\end{solution}


\declarecommand{\Wa}{W_{\smle!splitN!}}
\declarecommand{\Wb}{W_{\smle!leftmost!}}
\declarecommand{\Wc}{W_{\smle!halves!}}
\declarecommand{\Wd}{W_{\smle!rebalance!}}
\begin{task}{2} %
  Give a recurrence that describes the work $W_{\smle{rebalance}}(n)$ of
  \smle{rebalance} in terms of the \emph{size} $n$ of the input tree.
  Show how to obtain a closed form for this recurrence; your closed form may
  involve a sum.  Use this closed form to give a tight big-O bound for
  $W_{\smle{rebalance}}(n)$.
\end{task}

\begin{solution}
TYPE YOUR SOLUTION HERE
\end{solution}


\declarecommand{\Wa}{W_{\smle!splitN!}}
\declarecommand{\Wb}{W_{\smle!leftmost!}}
\declarecommand{\Wc}{W_{\smle!halves!}}
\declarecommand{\Wd}{W_{\smle!rebalance!}}
\declarecommand{\Ka}{\makebox[2em]{\ensuremath{k_1+k_2n}}}
\declarecommand{\K}[1]{\makebox[1.5em]{\ensuremath{k_1+k_2\frac{n}{#1}}}}
\declarecommand{\Kl}[1]{\makebox[1.5em][l]{\ensuremath{k_1+k_2\frac{n}{#1}}}}
\declarecommand{\Kr}[1]{\makebox[1.5em][r]{\ensuremath{k_1+k_2\frac{n}{#1}}}}
\begin{task}{2} %
  How does the work of \smle{rebalance} change if you know that the input tree
  is \emph{roughly balanced}?  A tree is roughly balance if its height is
  $O(\log\,n)$.  Redo the calculations for $W_{\smle{rebalance}}(n)$ to
  incorporate this additional piece of information.
\end{task}

\begin{solution}
TYPE YOUR SOLUTION HERE
\end{solution}


\declarecommand{\Wa}{W_{\smle!splitN!}}
\declarecommand{\Wb}{W_{\smle!leftmost!}}
\declarecommand{\Wc}{W_{\smle!halves!}}
\declarecommand{\Wd}{W_{\smle!rebalance!}}
\declarecommand{\Ka}{\makebox[2em]{\ensuremath{k_1+k_2\log\,n}}}
\declarecommand{\K}[1]{\makebox[2.3em]{\ensuremath{k_1+k_2\log\frac{n}{#1}}}}
\declarecommand{\Kl}[1]{\makebox[1.5em][l]{\ensuremath{k_1+k_2\log\frac{n}{#1}}}}
\declarecommand{\Kr}[1]{\makebox[1.5em][r]{\ensuremath{k_1+k_2\log\frac{n}{#1}}}}
\begin{task}{2} %
  Give a recurrence that describes the span $S_{\smle{splitN}}(d)$ of
  \smle{splitN} in terms of the \emph{height} $d$ of the input tree.  Give a
  tight big-O bound for $S_{\smle{splitN}}(d)$.
\end{task}

\begin{solution}
TYPE YOUR SOLUTION HERE
\end{solution}


\declarecommand{\Sa}{S_{\smle!splitN!}}
\declarecommand{\Sb}{S_{\smle!leftmost!}}
\declarecommand{\Sc}{S_{\smle!halves!}}
\declarecommand{\Sd}{S_{\smle!rebalance!}}
\begin{task}{2} %
  Give a recurrence that describes the span $S_{\smle{leftmost}}(d)$ of
  \smle{leftmost}, in terms of the \emph{height} $d$ of the input tree.  Give a
  tight big-O bound for $S_{\smle{leftmost}}(d)$.
\end{task}

\begin{solution}
TYPE YOUR SOLUTION HERE
\end{solution}


\declarecommand{\Sa}{S_{\smle!splitN!}}
\declarecommand{\Sb}{S_{\smle!leftmost!}}
\declarecommand{\Sc}{S_{\smle!halves!}}
\declarecommand{\Sd}{S_{\smle!rebalance!}}
\begin{task}{2} %
  Give a recurrence that describes the span $S_{\smle{halves}}(d)$ of
  \smle{halves} in terms of the \emph{height} of the input tree.  Give a tight
  big-O bound for $S_{\smle{halves}}(d)$.
\end{task}

\begin{solution}
TYPE YOUR SOLUTION HERE
\end{solution}


\declarecommand{\Sa}{S_{\smle!splitN!}}
\declarecommand{\Sb}{S_{\smle!leftmost!}}
\declarecommand{\Sc}{S_{\smle!halves!}}
\declarecommand{\Sd}{S_{\smle!rebalance!}}
\begin{task}{2}
  Give a recurrence that describes the span $S_{\smle{rebalance}}(n)$ of
  \smle{rebalance} in terms of the \emph{size} $n$ of the input tree.
  Show how to obtain a closed form for this recurrence; your closed form may
  involve a sum.  Use this closed form to give a tight big-O bound for
  $S_{\smle{rebalance}}(n)$.
\end{task}

\begin{solution}
TYPE YOUR SOLUTION HERE
\end{solution}


\declarecommand{\Sa}{S_{\smle!splitN!}}
\declarecommand{\Sb}{S_{\smle!leftmost!}}
\declarecommand{\Sc}{S_{\smle!halves!}}
\declarecommand{\Sd}{S_{\smle!rebalance!}}
\declarecommand{\Ka}{k_1+k_2n}
\declarecommand{\K}[1]{k_1+k_2\frac{n}{#1}}
\begin{task}{2}
  How does the span of \smle{rebalance} change if you know that the input tree
  is roughly balanced?  Redo the calculations for $S_{\smle{rebalance}}(n)$ to
  incorporate this additional piece of information.
\end{task}

\begin{solution}
TYPE YOUR SOLUTION HERE
\end{solution}


\declarecommand{\Sa}{S_{\smle!splitN!}}
\declarecommand{\Sb}{S_{\smle!leftmost!}}
\declarecommand{\Sc}{S_{\smle!halves!}}
\declarecommand{\Sd}{S_{\smle!rebalance!}}
\declarecommand{\Ka}{k_1+k_2\log\,n}
\declarecommand{\K}[1]{k_1+k_2\log\frac{n}{#1}}
\end{document}
