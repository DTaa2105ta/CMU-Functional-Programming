%% Packages and Macros for 15-150

\usepackage{graphicx}
\usepackage{amsmath}
\usepackage{amssymb}
\usepackage{amsthm}
\usepackage{stmaryrd}
\usepackage[colorlinks=true,breaklinks]{hyperref}
\usepackage[pdftex,dvipsnames]{xcolor} 
\usepackage{cancel}
\usepackage{framed}
\usepackage{comment}
\usepackage[short]{datetime}
\usepackage{caption}
\usepackage{subcaption}
\usepackage{etoolbox}
\usepackage{xparse}

\usepackage{multicol}
\setlength{\columnseprule}{1pt}

\usepackage{soul}
\sethlcolor{yellow}

\usepackage[margin=2cm]{geometry}

\newtheorem{remark}{Remark}
\newtheorem{theorem}{Theorem}
\newtheorem{lemma}{Lemma}
\newtheorem{definition}{Definition}
\newtheorem{corollary}{Corollary}
\newtheorem{property}[theorem]{Property}
\newtheorem{proposition}[theorem]{Proposition}

% For quotes
\usepackage{csquotes}

% symbols
\newrobustcmd{\eeq}{\ensuremath{\cong}}

% Boxes used in proof template
\usepackage[colorinlistoftodos,prependcaption,textsize=footnotesize]{todonotes}
\newcommand{\expl}[2]{\todo[inline,linecolor=#1,backgroundcolor=#1!25,bordercolor=#1]{#2}}

\newcommand{\think}[1]{%
  \begin{framed}
    {\color{red}\textbf{Brain teaser:}} #1
  \end{framed}
}
\newcommand{\exercise}[1]{%
  \begin{framed}
    {\color{red}\textbf{Exercise:}} #1
  \end{framed}
}

% Markers for quantified variables (soundness and completeness
% lecture)
\newcommand{\eigenvar}[1]{\colorbox{red!50}{#1}}
\newcommand{\term}[1]{\colorbox{green!50}{#1}}

% Listings setting for SML code
\usepackage{listings}
\definecolor{eclipseBlue}{RGB}{42,0.0,255}
\definecolor{eclipseGreen}{RGB}{63,127,95}
\lstdefinelanguage{SML}{
  basicstyle=\small\ttfamily,
  captionpos=b,
  tabsize=2,
  columns=fixed,
  breaklines=true,
  showstringspaces=false,
  mathescape=true,
  %frame=l,
  numbers=none,
  upquote=true,
  numberstyle=\small\ttfamily,
  morekeywords= {
    EQUAL, GREATER, LESS, NONE, SOME, 
    abstraction, abstype, and, andalso, 
    array, as, before, bool, case, char, 
    datatype, do, else, end, eqtype, 
    exception, exn, false, fn, fun, 
    functor, handle, if, in, include, 
    infix, infixr, int, let, list, local, 
    nil, nonfix, not, o, of, op, open, 
    option, orelse, overload, print, 
    raise, real, rec, ref, sharing, sig, 
    signature, string, struct, structure, 
    substring, then, true, type, unit, val, 
    vector, where, while, with, withtype, word
  },
  morestring=[b]",
  morecomment=[s]{(*}{*)},
  stringstyle=\color{black},
  identifierstyle=\color{eclipseBlue},
  keywordstyle=\color{red},
  commentstyle=\color{eclipseGreen},
  escapeinside={!?}{?!}
}

% Code block
\lstnewenvironment{sml}[1][]{\lstset{language=SML,#1}}{}
% Inline code
\def\smle{\lstinline[language=SML]}

% So that smle works inside math mode
\usepackage{letltxmacro}
\newcommand*{\SavedLstInline}{}
\LetLtxMacro\SavedLstInline\smle
\DeclareRobustCommand*{\smle}{%
  \ifmmode
    \let\SavedBGroup\bgroup
    \def\bgroup{%
      \let\bgroup\SavedBGroup
      \hbox\bgroup
    }%
  \fi
  \SavedLstInline
}

% For red-black trees
\usepackage{tikz}
\usetikzlibrary{arrows}
\usetikzlibrary{shapes.geometric}
\tikzset{
  treenode/.style = {align=center, inner sep=0pt, text centered,
    font=\sffamily},
  arn_n/.style = {treenode, circle, white, font=\sffamily\bfseries, draw=black,
    fill=black, text width=1.5em},% arbre rouge noir, noeud noir
  arn_r/.style = {treenode, circle, red, draw=red,
    text width=1.5em, very thick},% arbre rouge noir, noeud rouge
  arn_x/.style = {treenode, rectangle, %draw=black,
    minimum width=0.5em, minimum height=0.5em},% arbre rouge noir, nil
  arn_t/.style = {treenode, regular polygon, regular polygon sides=3,
    fill=Sepia!20, draw=black, minimum size=4.0em}% arbre rouge noir, tree
}
