
\documentclass[11pt]{article}

\newcommand{\issolution}{true}
\newcommand{\declarecommand}[1]{\providecommand{#1}{}\renewcommand{#1}}

%% Packages and Macros for 15-150

\usepackage{graphicx}
\usepackage{amsmath}
\usepackage{amssymb}
\usepackage{amsthm}
\usepackage{stmaryrd}
\usepackage[colorlinks=true,breaklinks]{hyperref}
\usepackage[pdftex,dvipsnames]{xcolor} 
\usepackage{cancel}
\usepackage{framed}
\usepackage{comment}
\usepackage[short]{datetime}
\usepackage{caption}
\usepackage{subcaption}
\usepackage{etoolbox}
\usepackage{xparse}

\usepackage{multicol}
\setlength{\columnseprule}{1pt}

\usepackage{soul}
\sethlcolor{yellow}

\usepackage[margin=2cm]{geometry}

\newtheorem{remark}{Remark}
\newtheorem{theorem}{Theorem}
\newtheorem{lemma}{Lemma}
\newtheorem{definition}{Definition}
\newtheorem{corollary}{Corollary}
\newtheorem{property}[theorem]{Property}
\newtheorem{proposition}[theorem]{Proposition}

% For quotes
\usepackage{csquotes}

% symbols
\newrobustcmd{\eeq}{\ensuremath{\cong}}

% Boxes used in proof template
\usepackage[colorinlistoftodos,prependcaption,textsize=footnotesize]{todonotes}
\newcommand{\expl}[2]{\todo[inline,linecolor=#1,backgroundcolor=#1!25,bordercolor=#1]{#2}}

\newcommand{\think}[1]{%
  \begin{framed}
    {\color{red}\textbf{Brain teaser:}} #1
  \end{framed}
}
\newcommand{\exercise}[1]{%
  \begin{framed}
    {\color{red}\textbf{Exercise:}} #1
  \end{framed}
}

% Markers for quantified variables (soundness and completeness
% lecture)
\newcommand{\eigenvar}[1]{\colorbox{red!50}{#1}}
\newcommand{\term}[1]{\colorbox{green!50}{#1}}

% Listings setting for SML code
\usepackage{listings}
\definecolor{eclipseBlue}{RGB}{42,0.0,255}
\definecolor{eclipseGreen}{RGB}{63,127,95}
\lstdefinelanguage{SML}{
  basicstyle=\small\ttfamily,
  captionpos=b,
  tabsize=2,
  columns=fixed,
  breaklines=true,
  showstringspaces=false,
  mathescape=true,
  %frame=l,
  numbers=none,
  upquote=true,
  numberstyle=\small\ttfamily,
  morekeywords= {
    EQUAL, GREATER, LESS, NONE, SOME, 
    abstraction, abstype, and, andalso, 
    array, as, before, bool, case, char, 
    datatype, do, else, end, eqtype, 
    exception, exn, false, fn, fun, 
    functor, handle, if, in, include, 
    infix, infixr, int, let, list, local, 
    nil, nonfix, not, o, of, op, open, 
    option, orelse, overload, print, 
    raise, real, rec, ref, sharing, sig, 
    signature, string, struct, structure, 
    substring, then, true, type, unit, val, 
    vector, where, while, with, withtype, word
  },
  morestring=[b]",
  morecomment=[s]{(*}{*)},
  stringstyle=\color{black},
  identifierstyle=\color{eclipseBlue},
  keywordstyle=\color{red},
  commentstyle=\color{eclipseGreen},
  escapeinside={!?}{?!}
}

% Code block
\lstnewenvironment{sml}[1][]{\lstset{language=SML,#1}}{}
% Inline code
\def\smle{\lstinline[language=SML]}

% So that smle works inside math mode
\usepackage{letltxmacro}
\newcommand*{\SavedLstInline}{}
\LetLtxMacro\SavedLstInline\smle
\DeclareRobustCommand*{\smle}{%
  \ifmmode
    \let\SavedBGroup\bgroup
    \def\bgroup{%
      \let\bgroup\SavedBGroup
      \hbox\bgroup
    }%
  \fi
  \SavedLstInline
}

% For red-black trees
\usepackage{tikz}
\usetikzlibrary{arrows}
\usetikzlibrary{shapes.geometric}
\tikzset{
  treenode/.style = {align=center, inner sep=0pt, text centered,
    font=\sffamily},
  arn_n/.style = {treenode, circle, white, font=\sffamily\bfseries, draw=black,
    fill=black, text width=1.5em},% arbre rouge noir, noeud noir
  arn_r/.style = {treenode, circle, red, draw=red,
    text width=1.5em, very thick},% arbre rouge noir, noeud rouge
  arn_x/.style = {treenode, rectangle, %draw=black,
    minimum width=0.5em, minimum height=0.5em},% arbre rouge noir, nil
  arn_t/.style = {treenode, regular polygon, regular polygon sides=3,
    fill=Sepia!20, draw=black, minimum size=4.0em}% arbre rouge noir, tree
}

%% Macros for 15-150 assignments

\newcommand{\printdue}[1]{\vspace{-0.5cm}{\Large{\textbf{\hl{Due:} #1}}}}

\newcommand{\hint}[1]{\footnote{ {\it Hint:} #1}}

%% Task and solution environments

\newcounter{taskCounter}
\setcounter{taskCounter}{0}

\newcounter{taskBonusCounter}
\setcounter{taskBonusCounter}{0}

\newcounter{pointsCounter}
\setcounter{pointsCounter}{0}

\newcounter{pointsBonusCounter}
\setcounter{pointsBonusCounter}{0}

% Task numbering variables resets to zero each time section changes
\newcounter{taskNum}[section]
\setcounter{taskNum}{0}

% Written tasks
\newenvironment{task}[1]
{
  \addtocounter{taskCounter}{1}
  \addtocounter{pointsCounter}{#1}
  \addtocounter{taskNum}{1}
  \begin{framed}
  \noindent
  \textbf{Task \arabic{section}.\arabic{taskNum}} (#1 points)
}
{
  \end{framed}
}

% Code tasks (do not need answers in the pdf)
\newenvironment{codetask}[1]
{
  \addtocounter{taskCounter}{1}
  \addtocounter{pointsCounter}{#1}
  \addtocounter{taskNum}{1}
  \begin{framed}
  \noindent
  \textbf{Coding Task \arabic{section}.\arabic{taskNum}} (#1 points)
}
{
  \end{framed}
}

% Bonus tasks are highlighted
\newenvironment{bonustask}[1]
{
  \addtocounter{taskBonusCounter}{1}
  \addtocounter{pointsBonusCounter}{#1}
  \addtocounter{taskNum}{1}
  \begin{framed}
  \noindent
  \colorbox{yellow}{%
    \textbf{Bonus Task \arabic{section}.\arabic{taskNum}} (#1 points)
  }
}
{
  \end{framed}
}

% Bonus code tasks (do not need answers in the pdf)
\newenvironment{bonuscodetask}[1]
{
  \addtocounter{taskBonusCounter}{1}
  \addtocounter{pointsBonusCounter}{#1}
  \addtocounter{taskNum}{1}
  \begin{framed}
  \noindent
  \colorbox{yellow}{%
    \textbf{Bonus Coding Task \arabic{section}.\arabic{taskNum}} (#1 points)
  }
}
{
  \end{framed}
}

% Lab tasks
\newenvironment{labtask}[1]
{
  \addtocounter{taskCounter}{1}
  \addtocounter{pointsCounter}{#1}
  \addtocounter{taskNum}{1}
  \begin{framed}
  \noindent
  \textbf{Task \arabic{section}.\arabic{taskNum}} (#1 points)
}
{
  \end{framed}
}


\AtEndDocument{\clearpage
  \typeout{^^J^^J^^J******** Assignment summary ********}
  \typeout{^^J-- \arabic{taskCounter} tasks worth \arabic{pointsCounter} points}
  \ifthenelse{\value{taskBonusCounter} > 0}
  {
    \typeout{^^J--\arabic{taskBonusCounter} bonus tasks worth \arabic{pointsBonusCounter} bonus points}
  }{}
  \typeout{^^J^^J************************************^^J^^J^^J}
}


\definecolor{solutioncolor}{rgb}{0.5, 0.0, 0.0}

% Reference solution (colored)
\ifthenelse{\equal{\issolution}{true}}
{
  \newenvironment{refsol}
  {%
    \noindent%
    \fbox{\textcolor{solutioncolor}{\bf Solution \arabic{section}.\arabic{taskNum}}} 
    \begingroup\color{solutioncolor}
  }
  {
    \endgroup
  }
}
{
  \excludecomment{refsol}
}

% Student solution (no color)
\ifthenelse{\equal{\issolution}{true}}
{
  \newenvironment{solution}
  {%
    \noindent%
    {\bf Solution \arabic{section}.\arabic{taskNum}}
  }
  {
  }
  \newenvironment{labsol}[1]
  {%
    \begin{center}
    \begin{tabular}{|p{0.97\textwidth}|}
    \hline
    {\bf Solution \arabic{section}.\arabic{taskNum}} \\[#1]
  }
  {
    \\\hline
    \end{tabular}
    \end{center}
  }
}
{
  \excludecomment{solution}
}



\title{\textbf{15-150 Fall 2024\\Homework 02}}
\author{YOUR NAME HERE -- \texttt{YOUR ANDREWID HERE}}

\begin{document}
\maketitle

\declarecommand{\true}{\mathit{true}}
\declarecommand{\false}{\mathit{false}}
\declarecommand{\Empty}{\mathit{Empty}}
\declarecommand{\Leaf}{\mathit{Leaf}}
\declarecommand{\Node}{\mathit{Node}}
\declarecommand{\iEmpty}{\mathit{iEmpty}}
\declarecommand{\iLeaf}{\mathit{iLeaf}}
\declarecommand{\iNode}{\mathit{iNode}}
\declarecommand{\validate}{\mathit{validate}}
\declarecommand{\iSize}{\mathit{iSize}}
\declarecommand{\iH}{\mathit{iSize'}}
\declarecommand{\tilt}{\mathit{tiltLeft}}
\declarecommand{\Elt}{S}
\section{Instrumented Trees}

\subsection{Instrumented Trees}

\begin{task}{2} %
Give an inductive definition for the function $\iSize: \mathbb{TT}
\rightarrow \mathbb{N}$ that computes the size of its argument.  The
inductive clause should make two recursive calls.
\end{task}

\begin{solution}
TYPE YOUR SOLUTION HERE
\end{solution}


\begin{task}{2} %
Give an inductive definition of the function $\validate: \mathbb{TT}
\rightarrow \mathbb{B}$ such that $\validate(t)$ returns $\true$ if the
imbalance at every inner node of $t$ is correct, and $\false$ otherwise.
\end{task}

\begin{solution}
TYPE YOUR SOLUTION HERE
\end{solution}


\begin{task}{2} %
Give an inductive definition to the function $\iH: \mathbb{TT} \rightarrow
\mathbb{N}$ such that $\iH(t) = \iSize(t)$ for a valid instrumented tree
$t$.  Your definition of $\iH$ must be such that the evaluation of $\iH(t)$
makes \emph{at most one} recursive call at each node it visits.  It may not
use $\iSize$.
\end{task}

\begin{solution}
TYPE YOUR SOLUTION HERE
\end{solution}


\begin{task}{2} %
Give an inductive definition of the function $\tilt: \mathbb{TT}
\rightarrow \mathbb{TT}$ such that $\tilt(t)$ is the tree obtained from
$t$ by selectively swapping the left and right subtrees so that each inner
node has a non-positive imbalance.
\end{task}

\begin{solution}
TYPE YOUR SOLUTION HERE
\end{solution}


\subsection{Properties}

\begin{task}{8} %
Prove this property using an appropriate form of induction.
\end{task}

\begin{solution}
TYPE YOUR SOLUTION HERE
\end{solution}


\begin{task}{10} %
Prove this property using an appropriate form of induction.  You may assume
the following lemma without proof (but you need to cite its use):

\medskip
\noindent
\textbf{Lemma:} For all $t\in\mathbb{TT}$, $\iSize(\tilt(t)) = \iSize(t)$.
\end{task}

\begin{solution}
TYPE YOUR SOLUTION HERE
\end{solution}


\subsection{Implementation}

\addtocounter{taskNum}{1}
\addtocounter{taskNum}{1}
\addtocounter{taskNum}{1}
\addtocounter{taskNum}{1}
\addtocounter{taskNum}{1}
\declarecommand{\Nil}{\mathit{nil}}
\declarecommand{\Cons}[2]{#1::#2}
\declarecommand{\TurnString}{right}
\declarecommand{\makeTurnString}{turnRight}
\declarecommand{\DOM}{\mathbb{N}}
\declarecommand{\Turn}{\mathit{\TurnString}}
\declarecommand{\Step}[1]{\mathit{step}\:#1}
\declarecommand{\N}{\mathit{north}}
\declarecommand{\E}{\mathit{east}}
\declarecommand{\W}{\mathit{west}}
\declarecommand{\makeTurn}{\mathit{\makeTurnString}}
\declarecommand{\Move}{\mathit{move}}
\declarecommand{\getPos}{\mathit{getPosition}}
\declarecommand{\reverse}{\mathit{reverse}}
\declarecommand{\goBack}{\mathit{goBack}}
\section{The SML-Robot}

\subsection{Operating the Robot}

\begin{task}{2} %
Define the function $\makeTurn: \mathbb{P} \rightarrow \mathbb{P}$ such that
$\makeTurn(p)$ returns the new position of the robot at position $p$ when
receiving the instruction $\Turn$.
\end{task}

\begin{solution}
TYPE YOUR SOLUTION HERE
\end{solution}


\begin{task}{2} %
Define the function $\Move: \DOM \times \mathbb{P} \rightarrow
\mathbb{P}$ such that $\Move(n, p)$ returns the new position of the robot at
position $p$ when receiving the instruction $\Step{n}$.
\end{task}

\begin{solution}
TYPE YOUR SOLUTION HERE
\end{solution}


\begin{task}{3} %
Define the function $\getPos: \mathbb{L_I} \times \mathbb{P} \rightarrow
\mathbb{P}$ such that, given itinerary $l$ and initial robot position $p$, the
function invocation $\getPos(l, p)$ returns its final position.  For example
(abbreviating $\Turn$ as
\ifthenelse{\equal{\TurnString}{left}}{``$L$''}{``$R$''} and $\Step{n}$ as
``$s\:n$'' for conciseness),
$$\renewcommand{\Step}[1]{s\:#1}
\ifthenelse{\equal{\TurnString}{left}}
{
\getPos([\Step{2}, L, L, L,
         \Step{2}, L,
         \Step{2}, L, L, L,
         \Step{2}, L, L, L,
         \Step{3}, L,
         \Step{1}],
         (0,0,\N))
}{
\getPos([\Step{2}, R,
         \Step{2}, R, R, R,
         \Step{2}, R,
         \Step{2}, R,
         \Step{3}, R, R, R,
         \Step{1}],
         (0,0,\N))
}
$$
has value $(5,1,\E)$.  This is visualized in Figure~\ref{fig:robot}.
\end{task}

\begin{solution}
TYPE YOUR SOLUTION HERE
\end{solution}


\subsection{Properties}

\begin{task}{6} %
Prove this property using an appropriate form of induction.  In your proof,
you may (of course) rely on the definition of \emph{append} (written $@$).
\end{task}

\begin{solution}
TYPE YOUR SOLUTION HERE
\end{solution}


\begin{task}{2} %
Define the function $\reverse: \mathbb{L_I} \rightarrow \mathbb{L_I}$ so that
$\reverse(l)$ returns the itinerary that reverses the moves in $l$, ignoring
the initial and final turnarounds.  For example,
$$
\reverse([\Step{2},\; \Turn,\; \Step{1}])
$$
returns the itinerary
$$
[\Step{1},\; \Turn,\; \Turn,\; \Turn,\; \Step{2}]
$$
(Convince yourself that this is indeed correct!)
\end{task}

\begin{solution}
TYPE YOUR SOLUTION HERE
\end{solution}


\begin{task}{1} %
Using $\reverse$, define the function $\goBack: \mathbb{L_I}
\rightarrow \mathbb{L_I}$ so that $\goBack(l)$ returns the itinerary that
allows the robot to go back to the position it was at before executing $l$.
Therefore,
$$
\goBack([\Step{2},\; \Turn,\; \Step{1}])
$$
returns the itinerary
$$
[\Turn,\; \Turn,\;
 \Step{1},\;
 \Turn,\; \Turn,\; \Turn,\;
 \Step{2},\;
 \Turn,\; \Turn]
$$
\end{task}

\begin{solution}
TYPE YOUR SOLUTION HERE
\end{solution}


\begin{task}{3} %
What is the corresponding property for $\goBack$?  State it and prove it
\end{task}

\begin{solution}
TYPE YOUR SOLUTION HERE
\end{solution}


\subsection{Implementation}

\addtocounter{taskNum}{1}
\addtocounter{taskNum}{1}
\addtocounter{taskNum}{1}
\addtocounter{taskNum}{1}
\addtocounter{taskNum}{1}
\addtocounter{taskNum}{1}
\end{document}
