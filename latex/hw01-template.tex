
\documentclass[11pt]{article}

\newcommand{\issolution}{true}
\newcommand{\declarecommand}[1]{\providecommand{#1}{}\renewcommand{#1}}

%% Packages and Macros for 15-150

\usepackage{graphicx}
\usepackage{amsmath}
\usepackage{amssymb}
\usepackage{amsthm}
\usepackage{stmaryrd}
\usepackage[colorlinks=true,breaklinks]{hyperref}
\usepackage[pdftex,dvipsnames]{xcolor} 
\usepackage{cancel}
\usepackage{framed}
\usepackage{comment}
\usepackage[short]{datetime}
\usepackage{caption}
\usepackage{subcaption}
\usepackage{etoolbox}
\usepackage{xparse}

\usepackage{multicol}
\setlength{\columnseprule}{1pt}

\usepackage{soul}
\sethlcolor{yellow}

\usepackage[margin=2cm]{geometry}

\newtheorem{remark}{Remark}
\newtheorem{theorem}{Theorem}
\newtheorem{lemma}{Lemma}
\newtheorem{definition}{Definition}
\newtheorem{corollary}{Corollary}
\newtheorem{property}[theorem]{Property}
\newtheorem{proposition}[theorem]{Proposition}

% For quotes
\usepackage{csquotes}

% symbols
\newrobustcmd{\eeq}{\ensuremath{\cong}}

% Boxes used in proof template
\usepackage[colorinlistoftodos,prependcaption,textsize=footnotesize]{todonotes}
\newcommand{\expl}[2]{\todo[inline,linecolor=#1,backgroundcolor=#1!25,bordercolor=#1]{#2}}

\newcommand{\think}[1]{%
  \begin{framed}
    {\color{red}\textbf{Brain teaser:}} #1
  \end{framed}
}
\newcommand{\exercise}[1]{%
  \begin{framed}
    {\color{red}\textbf{Exercise:}} #1
  \end{framed}
}

% Markers for quantified variables (soundness and completeness
% lecture)
\newcommand{\eigenvar}[1]{\colorbox{red!50}{#1}}
\newcommand{\term}[1]{\colorbox{green!50}{#1}}

% Listings setting for SML code
\usepackage{listings}
\definecolor{eclipseBlue}{RGB}{42,0.0,255}
\definecolor{eclipseGreen}{RGB}{63,127,95}
\lstdefinelanguage{SML}{
  basicstyle=\small\ttfamily,
  captionpos=b,
  tabsize=2,
  columns=fixed,
  breaklines=true,
  showstringspaces=false,
  mathescape=true,
  %frame=l,
  numbers=none,
  upquote=true,
  numberstyle=\small\ttfamily,
  morekeywords= {
    EQUAL, GREATER, LESS, NONE, SOME, 
    abstraction, abstype, and, andalso, 
    array, as, before, bool, case, char, 
    datatype, do, else, end, eqtype, 
    exception, exn, false, fn, fun, 
    functor, handle, if, in, include, 
    infix, infixr, int, let, list, local, 
    nil, nonfix, not, o, of, op, open, 
    option, orelse, overload, print, 
    raise, real, rec, ref, sharing, sig, 
    signature, string, struct, structure, 
    substring, then, true, type, unit, val, 
    vector, where, while, with, withtype, word
  },
  morestring=[b]",
  morecomment=[s]{(*}{*)},
  stringstyle=\color{black},
  identifierstyle=\color{eclipseBlue},
  keywordstyle=\color{red},
  commentstyle=\color{eclipseGreen},
  escapeinside={!?}{?!}
}

% Code block
\lstnewenvironment{sml}[1][]{\lstset{language=SML,#1}}{}
% Inline code
\def\smle{\lstinline[language=SML]}

% So that smle works inside math mode
\usepackage{letltxmacro}
\newcommand*{\SavedLstInline}{}
\LetLtxMacro\SavedLstInline\smle
\DeclareRobustCommand*{\smle}{%
  \ifmmode
    \let\SavedBGroup\bgroup
    \def\bgroup{%
      \let\bgroup\SavedBGroup
      \hbox\bgroup
    }%
  \fi
  \SavedLstInline
}

% For red-black trees
\usepackage{tikz}
\usetikzlibrary{arrows}
\usetikzlibrary{shapes.geometric}
\tikzset{
  treenode/.style = {align=center, inner sep=0pt, text centered,
    font=\sffamily},
  arn_n/.style = {treenode, circle, white, font=\sffamily\bfseries, draw=black,
    fill=black, text width=1.5em},% arbre rouge noir, noeud noir
  arn_r/.style = {treenode, circle, red, draw=red,
    text width=1.5em, very thick},% arbre rouge noir, noeud rouge
  arn_x/.style = {treenode, rectangle, %draw=black,
    minimum width=0.5em, minimum height=0.5em},% arbre rouge noir, nil
  arn_t/.style = {treenode, regular polygon, regular polygon sides=3,
    fill=Sepia!20, draw=black, minimum size=4.0em}% arbre rouge noir, tree
}

%% Macros for 15-150 assignments

\newcommand{\printdue}[1]{\vspace{-0.5cm}{\Large{\textbf{\hl{Due:} #1}}}}

\newcommand{\hint}[1]{\footnote{ {\it Hint:} #1}}

%% Task and solution environments

\newcounter{taskCounter}
\setcounter{taskCounter}{0}

\newcounter{taskBonusCounter}
\setcounter{taskBonusCounter}{0}

\newcounter{pointsCounter}
\setcounter{pointsCounter}{0}

\newcounter{pointsBonusCounter}
\setcounter{pointsBonusCounter}{0}

% Task numbering variables resets to zero each time section changes
\newcounter{taskNum}[section]
\setcounter{taskNum}{0}

% Written tasks
\newenvironment{task}[1]
{
  \addtocounter{taskCounter}{1}
  \addtocounter{pointsCounter}{#1}
  \addtocounter{taskNum}{1}
  \begin{framed}
  \noindent
  \textbf{Task \arabic{section}.\arabic{taskNum}} (#1 points)
}
{
  \end{framed}
}

% Code tasks (do not need answers in the pdf)
\newenvironment{codetask}[1]
{
  \addtocounter{taskCounter}{1}
  \addtocounter{pointsCounter}{#1}
  \addtocounter{taskNum}{1}
  \begin{framed}
  \noindent
  \textbf{Coding Task \arabic{section}.\arabic{taskNum}} (#1 points)
}
{
  \end{framed}
}

% Bonus tasks are highlighted
\newenvironment{bonustask}[1]
{
  \addtocounter{taskBonusCounter}{1}
  \addtocounter{pointsBonusCounter}{#1}
  \addtocounter{taskNum}{1}
  \begin{framed}
  \noindent
  \colorbox{yellow}{%
    \textbf{Bonus Task \arabic{section}.\arabic{taskNum}} (#1 points)
  }
}
{
  \end{framed}
}

% Bonus code tasks (do not need answers in the pdf)
\newenvironment{bonuscodetask}[1]
{
  \addtocounter{taskBonusCounter}{1}
  \addtocounter{pointsBonusCounter}{#1}
  \addtocounter{taskNum}{1}
  \begin{framed}
  \noindent
  \colorbox{yellow}{%
    \textbf{Bonus Coding Task \arabic{section}.\arabic{taskNum}} (#1 points)
  }
}
{
  \end{framed}
}

% Lab tasks
\newenvironment{labtask}[1]
{
  \addtocounter{taskCounter}{1}
  \addtocounter{pointsCounter}{#1}
  \addtocounter{taskNum}{1}
  \begin{framed}
  \noindent
  \textbf{Task \arabic{section}.\arabic{taskNum}} (#1 points)
}
{
  \end{framed}
}


\AtEndDocument{\clearpage
  \typeout{^^J^^J^^J******** Assignment summary ********}
  \typeout{^^J-- \arabic{taskCounter} tasks worth \arabic{pointsCounter} points}
  \ifthenelse{\value{taskBonusCounter} > 0}
  {
    \typeout{^^J--\arabic{taskBonusCounter} bonus tasks worth \arabic{pointsBonusCounter} bonus points}
  }{}
  \typeout{^^J^^J************************************^^J^^J^^J}
}


\definecolor{solutioncolor}{rgb}{0.5, 0.0, 0.0}

% Reference solution (colored)
\ifthenelse{\equal{\issolution}{true}}
{
  \newenvironment{refsol}
  {%
    \noindent%
    \fbox{\textcolor{solutioncolor}{\bf Solution \arabic{section}.\arabic{taskNum}}} 
    \begingroup\color{solutioncolor}
  }
  {
    \endgroup
  }
}
{
  \excludecomment{refsol}
}

% Student solution (no color)
\ifthenelse{\equal{\issolution}{true}}
{
  \newenvironment{solution}
  {%
    \noindent%
    {\bf Solution \arabic{section}.\arabic{taskNum}}
  }
  {
  }
  \newenvironment{labsol}[1]
  {%
    \begin{center}
    \begin{tabular}{|p{0.97\textwidth}|}
    \hline
    {\bf Solution \arabic{section}.\arabic{taskNum}} \\[#1]
  }
  {
    \\\hline
    \end{tabular}
    \end{center}
  }
}
{
  \excludecomment{solution}
}



\title{\textbf{15-150 Fall 2024\\Homework 01}}
\author{YOUR NAME HERE -- \texttt{YOUR ANDREWID HERE}}

\begin{document}
\maketitle

\section{Interpreting error messages}

\begin{task}{2} %
What error message do you see when you evaluate the \smle{errors.sml} file
without modifying it?  What caused this error? How can it be
fixed?
\end{task}

\begin{solution}
TYPE YOUR SOLUTION HERE
\end{solution}


\begin{task}{2} %
What is the next batch of errors?  What caused these errors? How can they be
fixed?
\end{task}

\begin{solution}
TYPE YOUR SOLUTION HERE
\end{solution}


\begin{task}{2} %
What error do you see after that?  What caused it? How can it be fixed?
\end{task}

\begin{solution}
TYPE YOUR SOLUTION HERE
\end{solution}


\begin{task}{2} %
What is the next error?  What caused it?  How can it be fixed?
\end{task}

\begin{solution}
TYPE YOUR SOLUTION HERE
\end{solution}


\begin{task}{2} %
After this, you should see two errors.  What are they?  What caused them?  How
can they be fixed?
\end{task}

\begin{solution}
TYPE YOUR SOLUTION HERE
\end{solution}


\begin{task}{4} %
Once you have fixed these errors, you will get a bunch more errors.  Two
simple fixes are sufficient to make them go away.  What are these errors? What
caused them? What are the fixes?
\end{task}

\begin{solution}
TYPE YOUR SOLUTION HERE
\end{solution}


\section{Specs and Functions}

\begin{task}{2}
\begin{sml}
(* octal n ==> r
 * REQUIRES: true
 * ENSURES:  r <> 0 (r is a non-zero number)
 *)
\end{sml}
\end{task}

\begin{solution}
TYPE YOUR SOLUTION HERE
\end{solution}


\begin{task}{2}
\begin{sml}
(* octal n ==> r
 * REQUIRES: n > 0
 * ENSURES:  true
 *)
\end{sml}
\end{task}

\begin{solution}
TYPE YOUR SOLUTION HERE
\end{solution}


\begin{task}{2}
\begin{sml}
(* octal n ==> r
 * REQUIRES: n >= 0
 * ENSURES:  r <> 0 (r is a non-zero number)
 *)
\end{sml}
\end{task}

\begin{solution}
TYPE YOUR SOLUTION HERE
\end{solution}


\begin{task}{2}
\begin{sml}
(* octal n ==> r
 * REQUIRES: n >= 0
 * ENSURES:  r >= 0 (r is a non-negative number)
 *)
\end{sml}
\end{task}

\begin{solution}
TYPE YOUR SOLUTION HERE
\end{solution}


\begin{task}{2}
\begin{sml}
(* octal n ==> r
 * REQUIRES: n >= 0
 * ENSURES:  r > 0 (r is a positive number)
 *)
\end{sml}
\end{task}

\begin{solution}
TYPE YOUR SOLUTION HERE
\end{solution}


\begin{task}{2} %
Which \emph{one} of these specifications gives the \emph{most} information
about the applicative behavior of the function \smle{octal}? Say
why, briefly.
\end{task}

\begin{solution}
TYPE YOUR SOLUTION HERE
\end{solution}


\section{Inductive Definitions}

\subsection{The Hosoya triangle}

\begin{task}{8} %
Give an inductive definition of the \emph{Hosoya coefficient} $H(n,k)$
\textbf{based on the way the Hosoya triangle is built}.  Consider carefully
the number of base and inductive cases, and the conditions under which each
applies.
\end{task}

\begin{solution}
TYPE YOUR SOLUTION HERE
\end{solution}


\begin{task}{5} %
The mathematical property defining that the left edge of the Hosoya triangle
corresponds to the Fibonacci number sequence is as follows:

\begin{property}
\label{prop:left-edge}
For all natural number $n$,
$$
H(n,0) = Fib(n)
$$
\end{property}
Prove this property by induction on $n$.
\end{task}

\begin{solution}
TYPE YOUR SOLUTION HERE
\end{solution}


\begin{task}{10} %
By induction on $n$, prove the following property:

\begin{property}
\label{prop:h-fib}
  For all natural numbers $n$ and $k$ such that $0 \leq k \leq n$,
  $$
  H(n,k) = Fib(k) \times Fib(n-k)
  $$
\end{property}
\textbf{Hint:} your  proof should match the structure of your inductive
definition for $H$ exactly: same cases, same conditions.
\end{task}

\begin{solution}
TYPE YOUR SOLUTION HERE
\end{solution}


\begin{task}{3} %
You may have noticed that each row of the Hosoya triangle has the same numbers
whether read from left to right or from right to left (the Hosoya triangle has
vertical symmetry).  This is mathematically expressed by the following
property:

\begin{property}
\label{prop:symmetry}
  For all natural numbers $n$ and $k$ such that $0 \leq k \leq n$,
  $$
  H(n, k) = H(n, n-k)
  $$
\end{property}

Prove this property.
\end{task}

\begin{solution}
TYPE YOUR SOLUTION HERE
\end{solution}


\begin{task}{3} %
Symmetry (or an easy observation) allows us to conclude that the rightmost
edge of the Hosoya triangle is again the Fibonacci numbers.  Formulate this
fact as a mathematical property and prove it.
\end{task}

\begin{solution}
TYPE YOUR SOLUTION HERE
\end{solution}


\subsection{Coding it up}

\addtocounter{taskNum}{1}
\end{document}
